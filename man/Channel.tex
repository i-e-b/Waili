%
%	Channel Class
%
%   $Id: Channel.tex,v 4.14.2.3.2.1 1999/07/20 16:16:14 geert Exp $
% 
%   Copyright (C) 1996-1999 Department of Computer Science, K.U.Leuven, Belgium
% 
%   This program is free software; you can redistribute it and/or modify
%   it under the terms of the GNU General Public License as published by
%   the Free Software Foundation; either version 2 of the License, or
%   (at your option) any later version.
% 
%   This program is distributed in the hope that it will be useful,
%   but WITHOUT ANY WARRANTY; without even the implied warranty of
%   MERCHANTABILITY or FITNESS FOR A PARTICULAR PURPOSE.  See the
%   GNU General Public License for more details.
% 
%   You should have received a copy of the GNU General Public License
%   along with this program; if not, write to the Free Software
%   Foundation, Inc., 59 Temple Place, Suite 330, Boston, MA  02111-1307  USA
%

\begin{manpage}{\libtitle}{Channel}{$ $Revision: 4.14.2.3.2.1 $ $}

\subtitle{Name}
    Channel --- Generic channel class

% -----------------------------------------------------------------------------

\subtitle{Description}
    This class provides a low-level channel abstraction. A channel is a
    (rectangular) matrix containing one-valued pixels (of type
    \texttt{PixType}).

% -----------------------------------------------------------------------------

\subtitle{Declaration}
    \needsinclude{Channel.h}

    \name{Channel} is an abstract base class. No instances
    can be declared.  Different channel types are implemented through
    inheritance.

\function{Channel()}
    Create an empty channel.

\function{Channel(u\_int cols, u\_int rows, int offx = 0, int offy = 0)}
    Create a channel with given dimensions. \name{cols} and \name{rows} are the
    number of columns respectively rows, \name{offx} and \name{offy} are the
    offsets of the upper left pixel in the universal coordinate system.

\function{Channel(const~Channel\& channel)}
    Create a new channel by copying channel \name{channel}.

% -----------------------------------------------------------------------------

\subtitle{Public \\ Operations}

\function{u\_int GetCols(void)\ const}
\function{u\_int GetRows(void)\ const}
    Get the number of columns respectively rows of the channel.

\function{int GetOffsetX(void)\ const}
\function{int GetOffsetY(void)\ const}
    Get the offset of upper left pixel of the channel in the universal
    coordinate system.

% -----------------------------------------------------------------------------

\subtitle{Static \\ Operations}

\function{Channel$\ast$ CreateFromDescriptor(u\_int cols, u\_int rows,
	const~TransformDescriptor transform[], u\_int depth,
	int offsetx = 0, int offsety = 0)}
    Create a channel with given dimensions. \name{cols} and \name{rows} are the
    number of columns respectively rows, \name{offsetx} and \name{offsety} are
    the offsets of the upper left pixel in the universal coordinate system. The
    channel will be pretransformed using the transform descriptor
    \name{transform} with transform depth \name{depth}.

% -----------------------------------------------------------------------------

\subtitle{Virtual \\ Operations}

\function{void GetMask(u\_int\& maskx, u\_int\& masky)\ const}
    Get the coordinate masks for the offsets. A set bit in a mask corresponds
    to a bit in the offset that can't be choosen freely without retransforming
    the channel.

\function{u\_int GetDepth(void)\ const}
    Get the transform depth of the channel.

\function{double Psnr(const~Channel\& channel, PixType maxval = 255)\ const}
    Calculate the Peak Signal to Noise Ratio (in dB) between the current
    channel en channel \name{channel}. \name{maxval} is the Peak Signal value.

\function{u64$\ast$ FullHistogram(PixType\& min, PixType\& max,
				  u64\& numpixels) const}
    Create a histogram for the current channel. The lower histogram limit will
    be put in \name{min}, the upper limit in \name{max}. The number of analyzed
    pixels will be put in \name{numpixels}. The result is an array of length
    $\name{max}-\name{min}+1$ containing the occurrency counts.

\function{double Entropy(void) const}
    Calculate the first order entropy (Shannon-Weaver) for this channel,
    in bits per pixel.

\function{PixType\& \mbox{operator()}(u\_int c, u\_int r)}
\function{PixType \mbox{operator()}(u\_int c, u\_int r)\ const}
    Access the `pixel' at column \name{c} and row \name{r}. This may refer to
    a wavelet coefficient instead of a real pixel value if the channel is
    wavelet transformed.

\function{void Clear(void)}
    Clear all pixel values to zero.

\function{void Resize(u\_int cols, u\_int rows)}
    Change the number of columns and rows of the channel to
    \name{cols} respectively \name{rows}.

\function{Channel$\ast$ Clone(void)\ const}
    Make a copy of the current channel.

\function{int SetOffsetX(void)\ const}
\function{int SetOffsetY(void)\ const}
    Change the offset of the channel in the universal coordinate system. If
    you change the bits that are covered by the corresponding coordinate mask,
    the channel will be retransformed.

\function{Channel$\ast$ Crop(int x1, int y1, int x2, int y2)\ const}
    Get a rectangular part of the current channel, of which the upper left
    corner is positioned at (\name{x1}, \name{y1}), and the lower right corner
    at (\name{x2}, \name{y2}).

\function{void Merge(const~Channel\& channel)}
    Paste \name{channel} into the current channel. The paste position is
    determined by the offsets of \name{channel}.

\function{void Add(const~Channel\& channel)}
\function{void Subtract(const~Channel\& channel)}
    Add respectively subtract \name{channel} to (from) the current channel.
    Both channels must have the same number of columns, number of rows,
    offsets and structure.

\function{Channel$\ast$ Diff(const~Channel\& channel)\ const}
    This function returns the difference channel between the current channel
    and \name{channel}. Both channels must have the same number of columns,
    number of rows, offsets and structure.

\function{void Enhance(f32 m)}
    Enhance the channel by multiplying all pixel values with \name{m}.
    If the channel is lifted, then only its high-pass coefficients will
    be changed.

\function{void Enhance(int m, u\_int shift)}
    Enhance the channel by multiplying all pixel values with \name{m} and
    shifting the result \name{shift} binary positions to the right.
    If the channel is lifted, then only its high-pass coefficients will
    be changed.

\function{LChannel$\ast$ PushFwtStepCR(const~Wavelet\& wavelet)}
\function{LChannel$\ast$ PushFwtStepC(const~Wavelet\& wavelet)}
\function{LChannel$\ast$ PushFwtStepR(const~Wavelet\& wavelet)}
    Add one transform level, using the wavelet transform specified by
    \name{wavelet}. The transform can operate on both columns and rows
    (\name{PushFwtStepCR}), on the colums only (\name{PushFwtStepC}) or on the
    rows only (\name{PushFwtStepR}). Note that the current channel will be
    destroyed!!!

    This function can return the following values:
    \begin{itemize}
    \item[\texttt{NULL}] The operation wasn't sucessful because the maximum
    number of transform levels was already reached.
    \item[\texttt{this}] If the result is equal to the current channel, the
    operation was successful.
    \item[Else] The operation was successful, and the current channel
    should be deleted and replaced by the returned channel.
    \end{itemize}

\function{Channel$\ast$ PopFwtStep(void)}
    Remove one transform level. This is the inverse operation of the last
    \name{PushFwtStep$\ast$} operation.

\function{u64 Threshold(double threshold, int soft = 0)}
    Perform hard (\name{soft = 0}) or soft (\name{soft = 1}) thresholding with
    the thresholding value \name{threshold}. The number of pixels that had a
    value smaller than \name{threshold} is returned.

\function{int IsLifted(void)\ const}
    This function returns non-zero if the current channel is already lifted.
    \emph{This should be removed later!! The user doesn't need to know what's
    the internal representation of the channel!}

\function{Channel$\ast$ Scale(f32 s, const~Wavelet\& wavelet)}
    Scale the channel with scaling factor \name{scale}. Note that the current
    channel will be destroyed!!!

% -----------------------------------------------------------------------------

\subtitle{Protected \\ Operations}

\function{Channel$\ast$ UpScale(u\_int s, const~Wavelet\& wavelet)}
    Scale the current channel up. The applied scaling factor is the next
    power of 2 of \name{s} if \name{s} is not a power of 2 by itself.
    Note that the current channel will be destroyed!!!

% -----------------------------------------------------------------------------

\subtitle{Virtual \\ Protected \\ Operations}

\function{void Destroy(void)}
    Delete the contents of the channel and zero the number of columns and rows.

\function{Channel$\ast$ DownScale(u\_int s, const~Wavelet\& wavelet)}
    Scale the current channel down. The applied scaling factor is the previous
    power of 2 of \name{s} if \name{s} is not a power of 2 by itself. Note
    that the current channel will be destroyed!!!

% -----------------------------------------------------------------------------

\subtitle{Static \\ Protected \\ Operations}

\function{int GetEven(int len)}
\function{int GetOdd(int len)}
    Calculate the number of even respectively odd coefficients for a signal
    with length \name{len}.

% -----------------------------------------------------------------------------

\separator

\subtitle{Derived \\ Classes}
    These are derived classes of the generic \name{Channel} class that
    add support for non-transformed and wavelet transformed channels.

% -----------------------------------------------------------------------------

\separator

\subtitle{Name}
    NTChannel --- Class for a non-transformed channel.

% -----------------------------------------------------------------------------

\subtitle{Declaration}

\function{NTChannel()}
    Create an empty non-transformed channel.

\function{NTChannel(u\_int cols, u\_int rows, int offx = 0, int offy = 0)}
    Create a non-transformed channel with given dimensions. \name{cols} and
    \name{rows} are the number of columns respectively rows, \name{offx} and
    \name{offy} are the offsets of the upper left pixel in the universal
    coordinate system.

\function{NTChannel(const~NTChannel\& channel)}
    Create a new non-transformed channel by copying the non-transformed channel
    \name{channel}.

% -----------------------------------------------------------------------------

\subtitle{Public \\ Operations}

\function{void GetMinMax(PixType\& min, PixType\& max, u\_int smoothing = 0)
	  const}
    Return the minimum and maximum pixel values in \name{min} respectively
    \name{max}. \name{smoothing} defines the degree of neglection of
    extraordinary pixel values.
    \emph{Smoothing is not yet implemented}

\function{s32$\ast$ Correlate(const~NTChannel\& channel, u\_int diff)\ const}
    Calculate the correlation matrix between the current channel and
    \name{channel}. \name{diff} indicates the difference of resolution level
    between the current channel and the more coarse \name{channel}. Note that
    the returned correlation matrix has the same dimensions as the current
    channel.

\function{u64$\ast$ Histogram(PixType min, PixType max)}
    Create a histogram for the current channel. The lower histogram limit will
    be \name{min}, the upper limit will be \name{max}. The result is an array
    of length $\name{max}-\name{min}+1$ containing the occurrency counts.

\function{u64 ThresholdHard(u\_int threshold)}
\function{u64 ThresholdSoft(u\_int threshold)}
    Perform hard or soft thresholding with the thresholding value
    \name{threshold}. The number of pixels that had a value smaller than
    \name{threshold} is returned.

\function{u\_int OptimalGCVThreshold(void)\ const}
    Calculate the optimal soft thresholding value using a technique called
    Generalized Cross Validation. Note that the channel should contain at least
    1000 pixels to give a meaningful result.

\function{NTChannel$\ast$ DupliScale(u\_int s)\ const}
    Create a scaled version of the current channel by duplicating the
    original pixel values. \name{s} is the scaling factor.

% -----------------------------------------------------------------------------

\subtitle{Virtual \\ Operations}

\function{NTChannel$\ast$ Clone(void)\ const}
    Make a copy of the current channel.

\function{NTChannel$\ast$ Crop(int x1, int y1, int x2, int y2)\ const}
    Get a rectangular part of the current channel, of which the upper left
    corner is positioned at (\name{x1}, \name{y1}), and the lower right corner
    at (\name{x2}, \name{y2}).

\function{NTChannel$\ast$ Diff(const~Channel\& channel)\ const}
    This function returns the difference channel between the current channel
    and \name{channel}. Both channels must have the same number of columns,
    number of rows, offsets and structure.

\function{LChannel$\ast$ Fwt(const~TransformDescriptor transform[],
			     u\_int depth)}
    Transform the channel using the Fast Wavelet Transform. The two-dimensional
    wavelet transform will be applied to all channels independently. The type
    of wavelet transform is determined by the \name{transform} array and its
    length \name{depth}. Note that the current channel will be destroyed!!!

% -----------------------------------------------------------------------------

\subtitle{Protected \\ Operations}

\function{void Interpolate(f32 s)}
    Scale the channel using a linear interpolation scheme.

\function{double GCV(u\_int threshold)\ const}
    Calculate the GCV value of the channel for Generalized Cross Validation.


% -----------------------------------------------------------------------------

\subtitle{Virtual \\ Protected \\ Operations}

\function{NTChannel$\ast$ DownScale(u\_int s, const~Wavelet\& wavelet)}
    Scale the current channel down. The applied scaling factor is the previous
    power of 2 of \name{s} if \name{s} is not a power of 2 by itself.

% -----------------------------------------------------------------------------

\separator

\subtitle{Name}
    LChannel --- Class for a wavelet transformed channel.

% -----------------------------------------------------------------------------

\subtitle{Declaration}
    \name{LChannel} is an abstract base class. No instances can be declared.
    Different wavelet transformed channel types are implemented through
    inheritance.

\function{LChannel(const~LChannel\& channel)}
    Create a new channel by copying channel \name{channel}.

% -----------------------------------------------------------------------------

\subtitle{Public \\ Operations}

\function{Channel$\ast$\& operator[](SubBand band)}
\function{const~Channel$\ast$\& operator[](SubBand band)\ const}
    Get the subband of type \name{band}.
    \emph{Make sure the subband does exist!}

\function{TransformDescriptor$\ast$ GetTransform(void)}
    Get a transform descriptor array for all transform levels.

\function{int GetShift(SubBand band)}
    Get the number of steps (in base-$\sqrt{2}$!) the coefficients of subband
    \name{band} have to be shifted to the left to obtain their real values.

\function{NTChannel$\ast$ IFwt(void)}
    Transform the channel using the inverse fast wavelet transform. Note that
    the current channel will be destroyed!!!

% -----------------------------------------------------------------------------

\subtitle{Virtual \\ Operations}

\function{TransformType GetTransformType(void)\ const}
    Get the transform type for this transform level.

\function{LChannel$\ast$ Clone(void)\ const}
    Make a copy of the current channel.

\function{NTChannel$\ast$ IFwt(int x1, int y1, int x2, int y2)\ const}
    Perform recursively the inverse fast wavelet transform on the rows and
    columns of the channel determined by the upper left and lower right corner
    respectively (\name{x1}, \name{y1}) and (\name{x2}, \name{y2}).

\function{LChannel$\ast$ Crop(int x1, int y1, int x2, int y2)\ const}
    Get a rectangular part of the current channel, of which the upper left
    corner is positioned at (\name{x1}, \name{y1}), and the lower right corner
    at (\name{x2}, \name{y2}).

% -----------------------------------------------------------------------------

\subtitle{Protected \\ Operations}

\function{LChannel(const~Wavelet\& filter, u\_int numsubbands, u\_int cols = 0,
		   u\_int rows = 0)}
    Create a channel with given dimensions. \name{numsubbands} is the number of
    subbands, \name{cols} and \name{rows} are the number of columns
    respectively rows.

% -----------------------------------------------------------------------------

\subtitle{Virtual \\ Protected \\ Operations}

\function{NTChannel$\ast$ IFwtStep(void)}
    Perform one step of the inverse fast wavelet transform. Note that the
    current channel will be destroyed!!!

\function{void Lazy(const~NTChannel\& source)}
    Calculate the Lazy Wavelet Transform of \name{source} and store the result
    in the current channel.

\function{void ILazy(NTChannel\& dest)\ const}
    Calculate the inverse Lazy Wavelet Transform of the current channel and
    put the result in \name{dest}.

\function{void CakeWalk(void)}
\function{void ICakeWalk(void)}
    Perform the (inverse) `Cake Walk' operation on the current channel.

\function{LChannel$\ast$ Crop\_rec(int x1, int y1, int x2, int y2,
				   NTChannel$\ast$ top, NTChannel$\ast$ bottom,
				   NTChannel$\ast$ left,
				   NTChannel$\ast$ right) const}
    Get a rectangular part of the current channel, of which the upper left
    corner is positioned at (\name{x1}, \name{y1}), and the lower right corner
    at (\name{x2}, \name{y2}). \name{top}, \name{bottom}, \name{left} and
    \name{right} are the resulting borders in the LP-band of the higher
    resolution level which affect lower resolutions.

\function{void Merge\_rec(const Channel$\ast$ channel,
			  NTChannel$\ast$ top, NTChannel$\ast$ bottom,
			  NTChannel$\ast$ left, NTChannel$\ast$ right)}
    Paste \name{channel} into the current channel. The paste position is
    determined by the offsets of \name{channel}. \name{top}, \name{bottom},
    \name{left} and \name{right} are the resulting borders in the LP-band of
    the higher resolution level which affect lower resolutions.

% -----------------------------------------------------------------------------

\subtitle{Subband \\ Types}
    Valid subband types are
    \begin{center}\begin{tabular}{|l|l|}
    \hline
    LL & lowpass in both the vertical and the horizontal direction \\
    LH & lowpass in the vertical, highpass in the horizontal direction \\
    HL & highpass in the vertical, lowpass in the horizontal direction \\
    HH & highpass in both the vertical and the horizontal direction \\
    \hline
    \end{tabular}\end{center}

% -----------------------------------------------------------------------------

\separator

\subtitle{Name}
    LChannelCR --- Class for a wavelet transformed channel (both columns and
		   rows).

% -----------------------------------------------------------------------------

\subtitle{Declaration}

\function{LChannelCR(const~Wavelet\& filter)}
    Create an empty channel.

\function{LChannelCR(const~Wavelet\& filter, u\_int cols, u\_int rows,
		     int offx, int offy)}
    Create a channel with given dimensions. \name{cols} and \name{rows} are the
    number of columns respectively rows, \name{offx} and \name{offy} are the
    coordinates in the universal coordinate system.

\function{LChannelCR(const~LChannelCR\& channel)}
    Create a new channel by copying channel \name{channel}.

% -----------------------------------------------------------------------------

\subtitle{Public \\ Operations}

\function{u\_int GetClow(void)\ const}
\function{u\_int GetChigh(void)\ const}
\function{u\_int GetRlow(void)\ const}
\function{u\_int GetRhigh(void)\ const}
    Get the number of columns and rows in the low and high frequency subbands.

% -----------------------------------------------------------------------------

\subtitle{Virtual \\ Operations}

\function{LChannelCR$\ast$ Clone(void)\ const}
    Make a copy of the current channel.

\function{LChannelCR$\ast$ Diff(const~Channel\& channel)\ const}
    This function returns the difference channel between the current channel
    and \name{channel}. Both channels must have the same number of columns,
    number of rows, offsets and structure.

\function{LChannelCR$\ast$ Crop\_rec(int x1, int y1, int x2, int y2,
				     NTChannel$\ast$ top,
				     NTChannel$\ast$ bottom,
				     NTChannel$\ast$ left,
				     NTChannel$\ast$ right) const}
    Get a rectangular part of the current channel, of which the upper left
    corner is positioned at (\name{x1}, \name{y1}), and the lower right corner
    at (\name{x2}, \name{y2}). \name{top}, \name{bottom}, \name{left} and
    \name{right} are the resulting borders in the LP-band of the higher
    resolution level which affect lower resolutions.

% -----------------------------------------------------------------------------

\separator

\subtitle{Name}
    LChannelC --- Class for a wavelet transformed channel (columns only).

% -----------------------------------------------------------------------------

\subtitle{Declaration}

\function{LChannelC(const~Wavelet\& filter)}
    Create an empty channel.

\function{LChannelC(const~Wavelet\& filter, u\_int cols, u\_int rows,
		     int offx, int offy)}
    Create a channel with given dimensions. \name{cols} and \name{rows} are the
    number of columns respectively rows, \name{offx} and \name{offy} are the
    coordinates in the universal coordinate system.

\function{LChannelC(const~LChannelC\& channel)}
    Create a new channel by copying channel \name{channel}.

% -----------------------------------------------------------------------------

\subtitle{Public \\ Operations}

\function{u\_int GetRlow(void)\ const}
\function{u\_int GetRhigh(void)\ const}
    Get the number of rows in the low and high frequency subbands.

% -----------------------------------------------------------------------------

\subtitle{Virtual \\ Operations}

\function{LChannelC$\ast$ Clone(void)\ const}
    Make a copy of the current channel.

\function{LChannelC$\ast$ Diff(const~Channel\& channel)\ const}
    This function returns the difference channel between the current channel
    and \name{channel}. Both channels must have the same number of columns,
    number of rows, offsets and structure.

\function{LChannelC$\ast$ Crop\_rec(int x1, int y1, int x2, int y2,
				    NTChannel$\ast$ top,
				    NTChannel$\ast$ bottom,
				    NTChannel$\ast$ left,
				    NTChannel$\ast$ right) const}
    Get a rectangular part of the current channel, of which the upper left
    corner is positioned at (\name{x1}, \name{y1}), and the lower right corner
    at (\name{x2}, \name{y2}). \name{top}, \name{bottom}, \name{left} and
    \name{right} are the resulting borders in the LP-band of the higher
    resolution level which affect lower resolutions.

% -----------------------------------------------------------------------------

\separator

\subtitle{Name}
    LChannelR --- Class for a wavelet transformed channel (rows only).

% -----------------------------------------------------------------------------

\subtitle{Declaration}

\function{LChannelR(const~Wavelet\& filter)}
    Create an empty channel.

\function{LChannelR(const~Wavelet\& filter, u\_int cols, u\_int rows,
		     int offx, int offy)}
    Create a channel with given dimensions. \name{cols} and \name{rows} are the
    number of columns respectively rows, \name{offx} and \name{offy} are the
    coordinates in the universal coordinate system.

\function{LChannelR(const~LChannelR\& channel)}
    Create a new channel by copying channel \name{channel}.

% -----------------------------------------------------------------------------

\subtitle{Public \\ Operations}

\function{u\_int GetClow(void)\ const}
\function{u\_int GetChigh(void)\ const}
    Get the number of columns in the low and high frequency subbands.

% -----------------------------------------------------------------------------

\subtitle{Virtual \\ Operations}

\function{LChannelR$\ast$ Clone(void)\ const}
    Make a copy of the current channel.

\function{LChannelR$\ast$ Diff(const~Channel\& channel)\ const}
    This function returns the difference channel between the current channel
    and \name{channel}. Both channels must have the same number of columns,
    number of rows, offsets and structure.

\function{LChannelR$\ast$ Crop\_rec(int x1, int y1, int x2, int y2,
				    NTChannel$\ast$ top,
				    NTChannel$\ast$ bottom,
				    NTChannel$\ast$ left,
				    NTChannel$\ast$ right) const}
    Get a rectangular part of the current channel, of which the upper left
    corner is positioned at (\name{x1}, \name{y1}), and the lower right corner
    at (\name{x2}, \name{y2}). \name{top}, \name{bottom}, \name{left} and
    \name{right} are the resulting borders in the LP-band of the higher
    resolution level which affect lower resolutions.

% -----------------------------------------------------------------------------

\separator

\subtitle{TransformDescriptor}
    The TransformDescriptor determines the kind of wavelet transform for one
    transform level. It contains 2 parts:

    \variable{TransformType type}
    \name{type} is the transform type and can be one of:
    \begin{center}\begin{tabular}{|l|l|}
    \hline
    \texttt{TT\_ColsRows} & Transform both columns and rows \\
    \texttt{TT\_Cols}     & Transform columns only \\
    \texttt{TT\_Rows}     & Transform rows only \\
    \hline
    \end{tabular}\end{center}

    \variable{const~Wavelet$\ast$ filter}
    \name{filter} is a pointer to a wavelet filter.

% -----------------------------------------------------------------------------

\subtitle{Dependency \\ Graphs}
    \begin{description}
    \item[Fig.~\ref{fig:Channel}] Inheritance dependency graph for the channel
    class hierarchy (\emph{Channel}).
    \end{description}

    \begin{figure}[ht]\begin{center}
    \resizebox{100mm}{!}{\includegraphics{Channel_dep.eps}}
    \caption{Inheritance dependency graph for the channel class hierarchy
    (\emph{Channel}).}
    \label{fig:Channel}
    \end{center}\end{figure}

\subtitle{See Also}
    The \name{Wavelet} and \name{Lifting} classes.

% -----------------------------------------------------------------------------

\subtitle{Revision}
     \rev{Channel.C,v 4.4.2.2 1999/04/15 09:35:09 geert Exp}
     \rev{Channel.h,v 4.5.2.3 1999/07/20 13:18:57 geert Exp}
     \rev{LChannel.C,v 4.5.2.2 1999/07/20 13:19:02 geert Exp}
     \rev{LChannel.h,v 4.3.2.2 1999/07/20 13:18:57 geert Exp}
     \rev{LChannelC.C,v 4.6.2.1 1999/07/20 13:19:02 geert Exp}
     \rev{LChannelC.h,v 4.3.2.1 1999/07/20 13:18:57 geert Exp}
     \rev{LChannelCR.C,v 4.6.2.1 1999/07/20 13:19:03 geert Exp}
     \rev{LChannelCR.h,v 4.3.2.1 1999/07/20 13:18:58 geert Exp}
     \rev{LChannelR.C,v 4.6.2.1 1999/07/20 13:19:04 geert Exp}
     \rev{LChannelR.h,v 4.3.2.1 1999/07/20 13:18:58 geert Exp}
     \rev{NTChannel.C,v 4.12.2.3 1999/07/20 13:19:04 geert Exp}
     \rev{NTChannel.h,v 4.8.2.1 1999/07/20 13:18:58 geert Exp}

\end{manpage}
