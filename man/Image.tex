%
%	Image Class
%
%   $Id: Image.tex,v 4.7.2.3.2.1 1999/07/20 16:16:15 geert Exp $
% 
%   Copyright (C) 1996-1999 Department of Computer Science, K.U.Leuven, Belgium
% 
%   This program is free software; you can redistribute it and/or modify
%   it under the terms of the GNU General Public License as published by
%   the Free Software Foundation; either version 2 of the License, or
%   (at your option) any later version.
% 
%   This program is distributed in the hope that it will be useful,
%   but WITHOUT ANY WARRANTY; without even the implied warranty of
%   MERCHANTABILITY or FITNESS FOR A PARTICULAR PURPOSE.  See the
%   GNU General Public License for more details.
% 
%   You should have received a copy of the GNU General Public License
%   along with this program; if not, write to the Free Software
%   Foundation, Inc., 59 Temple Place, Suite 330, Boston, MA  02111-1307  USA
%

\begin{manpage}{\libtitle}{Image}{$ $Revision: 4.7.2.3.2.1 $ $}

\subtitle{Name}
    Image --- Generic image class

% -----------------------------------------------------------------------------

\subtitle{Description}
    This class provides a low-level image abstraction. An image consists of a
    number of channels containing pixels (of type \texttt{PixType}).
    Each channel can have a different size. No interpretation or format is
    imposed on the channels and it's data.

% -----------------------------------------------------------------------------

\subtitle{Declaration}
    \needsinclude{Image.h}

\function{Image()}
    Create an empty image.

\function{Image(u\_int channels)}
    Create an empty image containing\name{channels} channels.

\function{Image(u\_int cols, u\_int rows, u\_int channels = 1)}
    Create an image with given dimensions. \name{cols} and \name{rows} are the
    number of columns respectively rows, \name{channels} is the number of
    channels. All channels will have the same size.

\function{Image(const~u\_int cols[], const~u\_int rows[], u\_int channels)}
    Create an image with given dimensions. \name{cols} and \name{rows} are
    arrays containing the number of columns respectively rows for every
    channel, \name{channels} is the number of channels.

\function{Image(const~Channel\& channel, u\_int channels = 1)}
    Create an image containing\name{channels} channels. Every channel will be a
    copy of \name{channel}.

\function{Image(const~Channel$\ast$ channel[], u\_int channels)}
    Create an image containing\name{channels} channels. The channels will be
    initialized using the array of channels \name{channel}.

\function{Image(const~Image\& im)}
    Create a new image by copying image \name{im}.

% -----------------------------------------------------------------------------

\subtitle{Public \\ Operations}

\function{ImageType Import(const~char$\ast$ filename,
			   ImageFormat format = IF\_AUTO)}
    Import an image from a file named \name{filename}, stored in a specific
    format \name{format}. If \name{format} is IF\_AUTO, the routine will try
    to guess the file format by examining the file name. The type of the image
    is returned. Images are assumed to contain 8-bit data, which is converted
    to the range $-128 \ldots 127$ for internal use.

\function{void Export(const~char$\ast$ filename,
		      ImageFormat format = IF\_AUTO)\ const}
    Export the to a file named \name{filename} using the specific file format
    \name{format}. If \name{format} is IF\_AUTO, the routine will try to guess
    the file format by examining the file name. The pixel values are considered
    to lay within the range $-128 \ldots 127$. If a pixel value doesn't fit, it
    will be clipped.

    Note: export in \name{IF\_TIFF} is not yet supported.

\function{void Convert(ImageType from, ImageType to)}
    Convert the image from type \name{from} to type \name{to}.
    \emph{Not all conversions are implemented yet.}

\function{u\_int GetChannels(void)\ const}
\function{u\_int GetCols(void)\ const}
\function{u\_int GetRows(void)\ const}
    Get the number of channels, columns or rows of the image.

\function{int GetOffsetX(void)\ const}
\function{int GetOffsetY(void)\ const}
    Get the offset of the first channel of the image in the universal
    coordinate system.

\function{Channel$\ast$\& operator[](u\_int channel)}
\function{const$\ast$~Channel\& operator[](u\_int channel)\ const}
    Access channel \name{channel}.

\function{PixType\& \mbox{operator()}(u\_int c, u\_int r, u\_int ch = 0)}
\function{const~PixType \mbox{operator()}(u\_int c, u\_int r, u\_int ch = 0)\
	  const}
    Access the `pixel' at column \name{c} and row \name{r} in channel
    \name{ch}. This may refer to a wavelet coefficient instead of a real pixel
    value if the channel is wavelet transformed.

\function{void Clear(void)}
    Clear all pixel values to zero.

\function{void Resize(u\_int cols, u\_int rows)}
    Change the number of columns and rows of the image to \name{cols}
    respectively \name{rows}. The number of channels is unchanged.

\function{void Resize(u\_int cols, u\_int rows, u\_int channels)}
    Change the number of columns, rows and channels of the image to
    \name{cols}, \name{rows} and \name{channels}. All channels will have the
    same size.

\function{Image\& operator$=$(const~Image\& im)}
    Make a copy of image \name{im}.

\function{Image$\ast$ Clone(void)\ const}
    Make a copy of the current image.

\function{void SetOffsetX(int offx)\ const}
\function{void SetOffsetY(int offy)\ const}
    Set the offset of the first channel of the image in the universal
    coordinate system.

\function{Image$\ast$ Crop(u\_int x1, u\_int y1, u\_int x2, u\_int y2)\ const}
    Cut the image so the upper left corner is positioned at
    (\name{x1}, \name{y1}), and the lower right corner at (\name{x2},
    \name{y2}).

\function{void Merge(const~Image\& im)}
    Paste image \name{im} into the current image. The paste position is
    determined by the offsets of \name{im}.

\function{void Add(const~Image\& im)}
\function{void Subtract(const~Image\& im)}
    Add respectively subtract image \name{im} to (from) the current image. Both
    images must have the same number of columns, rows and channels and their
    corresponding channels must have the same structure.

\function{Image$\ast$ Diff(const~Image\& im)\ const}
    This function returns the difference image between the current image and
    \name{im}. Both images must have the same number of columns, rows and
    channels and their corresponding channels must have the same structure.

\function{void InsertChannel(Channel\& data, u\_int ch)}
    Replace channel number \name{ch} of the image by the contents of channel
    \name{channel}.

\function{void DeleteChannel(u\_int channel)}
    Delete channel number \name{ch} from the image.

\function{void Fwt(const~TransformDescriptor transform[], u\_int depth)}
    Transform the image using the Fast Wavelet Transform. The two-dimensional
    wavelet transform will be applied to all channels independently. The type
    of wavelet transform is determined by the \name{transform} array and its
    length \name{depth}.

\function{void IFwt(void)}
    Transform the image using the inverse Fast Wavelet Transform. The
    two-dimensional inverse wavelet transform will be applied to all channels
    independently.  This is the inverse operation of \name{Fwt}.

\function{void Scale(f32 scale)}
    Scale the image with scaling factor \name{scale}.

% -----------------------------------------------------------------------------

\subtitle{Image Types \\ and Formats}
    The following image types are defined:
    \begin{center}\begin{tabular}{|l|l|}
    \hline
    \texttt{IT\_Unknown} & Unknown \\
    \texttt{IT\_Mono}    & Monochrome (black/white) \\
    \texttt{IT\_CIEY}    & Greyscale \\
    \texttt{IT\_CIEL}    & CIE luminance \\
    \texttt{IT\_RGB}     & RGB color \\
    \texttt{IT\_CIEXYZ}  & CIE XYZ color \\
    \texttt{IT\_CIELab}  & CIE L$^*$a$^*$b$^*$ color \\
    \texttt{IT\_YUV}     & YUV \\
    \texttt{IT\_YUVr}    & Reversible YUV \\
    \hline
    \end{tabular}\end{center}

    The following image formats are defined:
    \begin{center}\begin{tabular}{|l|l|}
    \hline
    \texttt{IF\_AUTO}     & Automatic \\
    \texttt{IF\_PNMASCII} & Portable AnyMap ASCII \\
    \texttt{IF\_PNMRAW}   & Portable AnyMap Binary \\
    \texttt{IF\_TIFF}     & Tag(ged) Image File Format \\
    \hline
    \end{tabular}\end{center}

% -----------------------------------------------------------------------------

\subtitle{See Also}
    The \name{Channel}, \name{Wavelet}, \name{Color} and \name{ColorSpace}
    classes.

% -----------------------------------------------------------------------------

\subtitle{Example}

\begin{verbatim}
//
//      Simple image compression example
//

#ifndef NULL
#define NULL    0
#endif

#include <waili/Image.h>

int main(void)
{
    const char infile[] = "image.pgm";
    const char outfile[] = "result.pgm";
    double threshold = 20.0;
    Image image;

    // Read the image
    image.Import(infile);

    // Transform the image using the Cohen-Daubechies-Feauveau
    // (2, 2) biorthogonal wavelets
    Wavelet *wavelet = Wavelet::CreateCDF(2, 2);
    TransformDescriptor transform[] = {
	{ TT_ColsRows, wavelet },
	{ TT_ColsRows, wavelet },
	{ TT_ColsRows, wavelet },
	{ TT_ColsRows, wavelet },
	{ TT_ColsRows, wavelet },
	{ TT_ColsRows, wavelet },
	{ TT_ColsRows, wavelet },
	{ TT_ColsRows, wavelet }
    };
    image.Fwt(transform, sizeof(transform)/sizeof(*transform));

    // Zero all entries smaller than the threshold
    for (u_int ch = 0; ch < image.GetChannels(); ch++)
	image[ch]->Threshold(threshold);

    // Inverse wavelet transform
    image.IFwt();

    // Write the reconstructed image to a file
    image.Export(outfile);
    return(0);
}
\end{verbatim}

% -----------------------------------------------------------------------------

\subtitle{Revision}
     \rev{Image.C,v 4.4.2.4 1999/07/20 13:19:02 geert Exp}
     \rev{Image.h,v 4.6.2.3 1999/07/20 13:18:57 geert Exp}
     \rev{Example.C,v 4.0.2.1 1998/06/22 13:49:10 geert Exp}

\end{manpage}
