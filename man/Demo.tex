%
%	A Simple Demo Program
%
%   $Id: Demo.tex,v 4.6.2.1.2.1 1999/07/20 16:16:15 geert Exp $
% 
%   Copyright (C) 1996-1999 Department of Computer Science, K.U.Leuven, Belgium
% 
%   This program is free software; you can redistribute it and/or modify
%   it under the terms of the GNU General Public License as published by
%   the Free Software Foundation; either version 2 of the License, or
%   (at your option) any later version.
% 
%   This program is distributed in the hope that it will be useful,
%   but WITHOUT ANY WARRANTY; without even the implied warranty of
%   MERCHANTABILITY or FITNESS FOR A PARTICULAR PURPOSE.  See the
%   GNU General Public License for more details.
% 
%   You should have received a copy of the GNU General Public License
%   along with this program; if not, write to the Free Software
%   Foundation, Inc., 59 Temple Place, Suite 330, Boston, MA  02111-1307  USA
%

\section{A simple demo program}

\texttt{Lifting/test/Demo} is a simple interactive demo program that allows you
to play with wavelet transforms. It understands the following commands:

\begin{description}
\item[\texttt{Help}]
\item[\texttt{?}]\mbox{}\\
    Display some help information.
\item[\texttt{Quit}]
\item[\texttt{Exit}]\mbox{}\\
    Terminate the program.
\item[\texttt{Load} \emph{image}]\mbox{}\\
    Load an image from file \emph{image}. Make sure this file does exist!
\item[\texttt{Save} \emph{image}]\mbox{}\\
    Save the current image to file \emph{image}.
\item[\texttt{View}]\mbox{}\\
    View the current image using \emph{xv}. Make sure the \texttt{xv}
    executable is in your path!
\item[\texttt{Wavelet} $n$ $\widetilde{n}$]\mbox{}\\
    Use the biorthogonal Cohen-Daubechies-Feauveau wavelet with $(n,
    \widetilde{n})$ vanishing moments. Make sure you select a supported
    wavelet!
\item[\texttt{Wavelet} $n$]\mbox{}\\
    Use a biorthogonal wavelet from the JPEG2000 draft. Values of $n$:
    \begin{description}
	\item[1] CRF (13, 7)
	\item[2] SWE (13, 7)
    \end{description}
\item[\texttt{Fstep cr}]
\item[\texttt{Fstep c}]
\item[\texttt{Fstep r}]\mbox{}\\
    Add one transform level. The transform can operate on both colums and rows
    (default), or on the columns or rows only.
\item[\texttt{Bstep}]\mbox{}\\
    Remove one transform level.
\item[\texttt{Ifwt}]\mbox{}\\
    Perform the full inverse transform.
\item[\texttt{Noise} \emph{var}]\mbox{}\\
    Add white Gaussian noise with variance \emph{var}.
\item[\texttt{Denoise}]\mbox{}\\
    Denoise the wavelet transformed image by using soft thresholding with a GCV
    (Generalized Cross Validation) estimated threshold. Only subbands that
    count at least 1000 pixels will be thresholded.
\item[\texttt{Backup}]\mbox{}\\
    Create a backup of the current image for later comparison.
\item[\texttt{Psnr}]\mbox{}\\
    Calculate the PSNR (Peak Signal to Noise Ratio) of the current image,
    compared to the backup image.
\item[\texttt{Threshold} \emph{value}]\mbox{}\\
    Perform hard thresholding with threshold value \emph{value}.
\item[\texttt{Scale} \emph{value}]\mbox{}\\
    Scale the image with factor \emph{value}.
\item[\texttt{Histogram} \emph{level subband channel}]\mbox{}\\
    View the histogram of subband \emph{subband} at level \emph{level} of the
    decomposition of channel \emph{channel}.
\item[\texttt{Entropy}]\mbox{}\\
    Calculate the first order entropy (Shannon-Weaver) for this channel,
    in bits per pixel.
\item[\texttt{Yuv}]\mbox{}\\
    Convert from RGB to YUVr (or vice versa).
\end{description}

All commands can be abbreviated.

\subsection*{Revision}

\rev{Demo.C,v 4.6.2.2 1999/04/15 10:10:14}
