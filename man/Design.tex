%
%	Design and Implementation of the wavelet transform library
%
%   $Id: Design.tex,v 4.5.2.3.2.1 1999/07/20 16:16:15 geert Exp $
% 
%   Copyright (C) 1996-1999 Department of Computer Science, K.U.Leuven, Belgium
% 
%   This program is free software; you can redistribute it and/or modify
%   it under the terms of the GNU General Public License as published by
%   the Free Software Foundation; either version 2 of the License, or
%   (at your option) any later version.
% 
%   This program is distributed in the hope that it will be useful,
%   but WITHOUT ANY WARRANTY; without even the implied warranty of
%   MERCHANTABILITY or FITNESS FOR A PARTICULAR PURPOSE.  See the
%   GNU General Public License for more details.
% 
%   You should have received a copy of the GNU General Public License
%   along with this program; if not, write to the Free Software
%   Foundation, Inc., 59 Temple Place, Suite 330, Boston, MA  02111-1307  USA
%

\section{Design and Implementation of \libname}

\libname\ is meant to operate on two-dimensional images of various kinds.
Applications are situated in image processing.


\subsection{Design decisions}

This section discusses some of the design decisions we made for this library.
For more information about the theoretical foundations behind the library,
please refer to `Wavelet Transforms Using the Lifting Scheme' (Report
ITA-Wavelets-WP1.1) \cite{ITA-Wavelets-WP1.1,UytVWuJanRooBul98}.

We chose to implement two-dimensional wavelet transforms using the integer
version of the Lifting Scheme. The wavelets we use are a subclass of the
Cohen-Daubechies-Feauveau family of biorthogonal wavelets.

\subsubsection{The Lifting Scheme}

The Lifting Scheme \cite{swe:lift1,swe:lift2,swe:spie95} provides a fast and
simple algorithm for arbitrary wavelet transforms \cite{ds:factor}. Furthermore
the inverse transform is trivial to find.

Although the Lifting Scheme allows to transform signals with a finite length
without extending the signal, we did not choose to take this approach. Instead
we use the classical symmetric extension \cite{bris:acha96} because it's
easier to implement and suffices for the applications we have in mind.


\subsubsection{The integer wavelet transform}

In many applications (e.g.\ image compression and processing) the input data
consists of integer samples only. In addition the storage and encoding of
integer numbers is easier, compared to floating point numbers.

To take advantage of this we use the integer version of the Lifting Scheme,
which maps integers to integers and is reversible, retaining the perfect
reconstruction property \cite{cdsy:integer}.

All arithmetic operations are done in 16 bit. This should suffice for
applications where the input data is 8 bit wide. Of course this can easily be
changed if necessary.


\subsubsection{Cohen-Daubechies-Feauveau biorthogonal wavelets}

The key benefits of the Cohen-Daubechies-Feauveau biorthogonal wavelets \cite{coh-dau-fea:biorthogonal} are:
\begin{itemize}
\item They have finite support. This preserves the locality of image features.
\item The scaling function $\phi(x)$ is always symmetric, and the wavelet
      function $\psi(x)$ is always symmetric or antisymmetric. This is
      important for image processing operations.
\item Its filter coefficients are of the form $z\over2^n$, with $z \in
      \mathbf{Z}$ and $n \in \mathbf{N}$. This simplifies the implementation.
      But unfortunately this feature isn't always preserved by the
      decomposition in lifting steps.
\end{itemize}

We choose not to use wavelets with more than 6 vanishing moments to restrict
the filter lengths. Longer filters have less locality and thus perform worse in
image processing applications, in spite of their increase in smoothness.

We implemented the following wavelet transforms of this family($(n, \tilde{n})$
means that the primal wavelet has $n$ vanishing moments, while the dual wavelet
has $\tilde{n}$ vanishing moments):
\begin{description}
\item[\boldmath{(1, x)}:] $(1, 1)$, $(1, 3)$, $(1, 5)$
\item[\boldmath{(2, x)}:] $(2, 2)$, $(2, 4)$, $(2, 6)$
\item[\boldmath{(4, x)}:] $(4, 2)$, $(4, 4)$, $(4, 6)$
\end{description}
We deliberately didn't implement any of the $(3, x)$ or $(5, x)$ wavelet
transforms because their lifting steps require divisions by 3 or 5, which are
not reversible in integer math. $(6, x)$ aren't implemented either because they
require more than 16 bits (for 8 bit input data).


\subsubsection{Wavelets and translation-invariance}

A disadvantage of the wavelet transform is that it's not translation-invariant:
if the image is translated before performing the wavelet transform, the result
is not a translated version of the wavelet transform of the original image.
The redundant wavelet transform is translation-invariant, but it needs much
more memory and processing time, so this isn't an option in many applications.

Since we wanted to allow crop and merge operations on wavelet transformed
images we came up with the following scheme.

If each transform level is considered independently, one step of a wavelet
transform is translation-invariant if the translation is limited to an even
number of pixels. Thus we associate with every matrix \emph{coordinates} (a
horizontal and vertical offset for the upper left pixel) which
depend on the transform level. At every transform level we have two versions of
the wavelet transform: an \emph{even} and an \emph{odd} version. Which
transform is used depends on the parity of the offset.

If the parities of the coordinates match at each level, we can merge two
images without retransforming one of them. If they don't match, we have to
retransform one image. The main idea behind this scheme is that in many cases
the coordinates of the subimage that will be pasted into another image are
known in advance, so it can be transformed correctly. An example of this is the
creation of one large image by concatenating several separately created
subimages.


\subsection{Implementation}

The software library is written in \CC. We extensively use features of the
ISO \CC\ Standard, which was finalized in November 1997 (from now on called
\emph{\CC\ 97}), since they provide a great enrichment of the \CC\ language and
allow for a cleaner design.

Unfortunately there aren't many compilers that adhere to \CC\ 97 yet. The
development was done using \emph{GNU \CC\ 2.7.2} and \emph{egcs 1.0}.
Fortunately these compilers are available for about any platform, and they're
free\footnote{Available from \texttt{ftp://prep.ai.mit.edu/pub/gnu/} and
\texttt{http://egcs.cygnus.com/}.}!


\subsubsection{\emph{Image} objects}

An \emph{Image} consists of one or more independent channels, thus allowing for
different sizes and wavelet transform types per channel. No interpretation or
format is imposed on the channels and its data. The actual meaning of the
image data can be freely choosen by the user. Examples are grayscale, RGB, YUV
or Lab color, etc.\ \ldots


\subsubsection{\emph{Channel} objects}

The basic building block of the library is the \emph{Channel}. A channel is a
rectangular matrix containing one-valued pixels. A channel can be
non-transformed (a \emph{NTChannel}), or wavelet-transformed (a
\emph{LChannel}\footnote{Rumors say that \emph{NT} and \emph{L} refer to
two popular operating systems --- with the goal of this project to convert as
many NTChannels to LChannels as possible --- but this hasn't been confirmed
officially.}).

Since a wavelet transform is some kind of \emph{recursive} transform, a
LChannel contains some subchannels (subbands), which can be either
non-transformed or wavelet-transformed. The number of subchannels in a LChannel
depends on the type of wavelet transform. You can have the following
combinations:

\begin{description}
\item[\emph{LChannelCR}] Obtained by transforming both the columns and
rows of a NTChannel. As a result, you have 4 subbands:
\begin{description}
\item[LL] Low pass band in both the horizontal and the vertical direction,
\item[LH] Low pass band in the vertical direction, high pass in the horizontal
	  direction,
\item[HL] High pass band in the vertical direction, low pass in the horizontal
	  direction,
\item[HH] High pass band in both the horizontal and the vertical direction.
\end{description}

\item[\emph{LChannelC}] Obtained by transforming only the columns of a
NTChannel. As a result, you have 2 subbands:
\begin{description}
\item[L] Low pass band in the vertical direction,
\item[H] High pass band in the vertical direction.
\end{description}

\item[\emph{LChannelR}] Obtained by transforming only the rows of a
NTChannel. As a result, you have 2 subbands:
\begin{description}
\item[L] Low pass band in the horizontal direction,
\item[H] High pass band in the horizontal direction.
\end{description}
\end{description}

Fig.~\ref{fig:eg_transform} shows an example of a channel after two transform
levels.

\begin{figure}\begin{center}
\resizebox{100mm}{!}{\includegraphics{Decomposition.eps}}
\caption{Example of a channel after two transform levels. In the first step
both the columns and the rows are transformed, in the second step only the
columns are transformed.}
\label{fig:eg_transform}
\end{center}\end{figure}


\subsubsection{\emph{Wavelet} objects}

A \emph{Wavelet} represents the filters and lifting steps associated with a
specific wavelet transform. Some wavelet transforms of the
Cohen-Daubechies-Feauveau family are implemented.

You can add your own favorite wavelet transform if you have a decomposition in
integer lifting steps for it.

