%
%	Lifting Operations
%
%   $Id: Lifting.tex,v 4.6.2.2.2.1 1999/07/20 16:16:16 geert Exp $
% 
%   Copyright (C) 1996-1999 Department of Computer Science, K.U.Leuven, Belgium
% 
%   This program is free software; you can redistribute it and/or modify
%   it under the terms of the GNU General Public License as published by
%   the Free Software Foundation; either version 2 of the License, or
%   (at your option) any later version.
% 
%   This program is distributed in the hope that it will be useful,
%   but WITHOUT ANY WARRANTY; without even the implied warranty of
%   MERCHANTABILITY or FITNESS FOR A PARTICULAR PURPOSE.  See the
%   GNU General Public License for more details.
% 
%   You should have received a copy of the GNU General Public License
%   along with this program; if not, write to the Free Software
%   Foundation, Inc., 59 Temple Place, Suite 330, Boston, MA  02111-1307  USA
%

\begin{manpage}{\libtitle}{Lifting}{$ $Revision: 4.6.2.2.2.1 $ $}

\subtitle{Name}
    Lifting --- Generic class for integer \name{Lifting Scheme} steps

% -----------------------------------------------------------------------------

\subtitle{Description}
    This class provides a generic lifting step interface, to be used for
    wavelet transforms using the \name{Lifting Scheme}.

% -----------------------------------------------------------------------------

\subtitle{Declaration}
    \needsinclude{Lifting.h}

    \name{Lifting} is an abstract base class. No instances can be declared.
    Lifting steps on different types of data are implemented through
    inheritance.

% -----------------------------------------------------------------------------

\subtitle{Virtual \\ Operations}

\function{void Lift\_L1R1\_FR(int primal, const~s16 b[2], const~u16 a)\ const}
\function{void ILift\_L1R1\_FR(int primal, const~s16 b[2], const~u16 a)\ const}
\function{void Lift\_L2R2\_FR(int primal, const~s16 b[4], const~u16 a)\ const}
\function{void ILift\_L2R2\_FR(int primal, const~s16 b[4], const~u16 a)\ const}
\function{void Lift\_L3R3\_FR(int primal, const~s16 b[6], const~u16 a)\ const}
\function{void ILift\_L3R3\_FR(int primal, const~s16 b[6], const~u16 a)\ const}
    Primal (\name{primal} $= 1$) and dual (\name{primal} $= 0$) integer lifting
    steps with full rounding. \name{Lift\_LmRn\_FR} implements a lifting
    operation of the form
    \[ x_i \leftarrow x_i + \left\{ {\sum_{j=-m}^{n-1} b_{j+m} y_j \over a}
       \right\}, \]
    with $x_i$ and $y_i$ the low pass and high pass samples (or vice versa,
    depending on the value of \name{primal}), and $\left\{\right\}$ a rounding
    operation.
    \name{ILift\_LmRn\_FR} is the corresponding inverse operation.

\function{void Lift\_L1R1\_NR(int primal, const~s16 b[2], const~u16 a)\ const}
\function{void ILift\_L1R1\_NR(int primal, const~s16 b[2], const~u16 a)\ const}
\function{void Lift\_L2R2\_NR(int primal, const~s16 b[4], const~u16 a)\ const}
\function{void ILift\_L2R2\_NR(int primal, const~s16 b[4], const~u16 a)\ const}
\function{void Lift\_L3R3\_NR(int primal, const~s16 b[6], const~u16 a)\ const}
\function{void ILift\_L3R3\_NR(int primal, const~s16 b[6], const~u16 a)\ const}
    Primal (\name{primal} $= 1$) and dual (\name{primal} $= 0$) integer lifting
    steps without rounding. \name{Lift\_LmRn\_FR} implements a lifting
    operation of the form
    \[ x_i \leftarrow a x_i + \sum_{j=-m}^{n-1} b_{j+m} y_j, \]
    with $x_i$ and $y_i$ the low pass and high pass samples (or vice versa,
    depending on the value of \name{primal}).
    \name{ILift\_LmRn\_NR} is the corresponding inverse operation.

\function{void Lift\_L1R1\_MX(int primal, const~s16 b[2], const~u16 a1,
			      const~u16 a2)\ const}
\function{void ILift\_L1R1\_MX(int primal, const~s16 b[2], const~u16 a1,
			       const~u16 a2)\ const}
\function{void Lift\_L2R2\_MX(int primal, const~s16 b[4], const~u16 a1,
			      const~u16 a2)\ const}
\function{void ILift\_L2R2\_MX(int primal, const~s16 b[4], const~u16 a1,
			       const~u16 a2)\ const}
\function{void Lift\_L3R3\_MX(int primal, const~s16 b[6], const~u16 a1,
			      const~u16 a2)\ const}
\function{void ILift\_L3R3\_MX(int primal, const~s16 b[6], const~u16 a1,
			       const~u16 a2)\ const}
    Primal ($\name{primal} = 1$) and dual (\name{primal} $= 0$) integer lifting
    steps with mixed rounding. \name{Lift\_LmRn\_FR} implements a lifting
    operation of the form
    \[ x_i \leftarrow a_1 x_i + \left\{ {\sum_{j=-m}^{n-1} b_{j+m} y_j \over
       a_2} \right\}, \]
    with $x_i$ and $y_i$ the low pass and high pass samples (or vice versa,
    depending on the value of \name{primal}), and $\left\{\right\}$ a rounding
    operation.
    \name{ILift\_LmRn\_MX} is the corresponding inverse operation.

% -----------------------------------------------------------------------------

\separator

\subtitle{Derived \\ Classes}
    Lifting operations on various objects are available through classes derived
    from the \name{Lifting} class:

% -----------------------------------------------------------------------------

\separator

\subtitle{Name}
    LiftChannelR --- Lifting operations on the rows of 2 \name{NTChannel}s

% -----------------------------------------------------------------------------

\subtitle{Declaration}

\function{LiftChannelR(NTChannel$\ast$ lowpass, NTChannel$\ast$ highpass)}
    Create a Lifting object for lifting operations on the rows of 2
    \name{NTChannels}.
    \name{lowpass} contains the low pass samples, while \name{highpass}
    contains the high pass samples. Both \name{lowpass} and \name{highpass}
    must have the same number of rows, and the number of columns of
    \name{lowpass} and \name{highpass} must differ maximum 1.

% -----------------------------------------------------------------------------

\separator

\subtitle{Name}
    LiftChannelC --- Lifting operations on the columns of 2 \name{NTChannel}s

% -----------------------------------------------------------------------------

\subtitle{Declaration}

\function{LiftChannelC(NTChannel$\ast$ lowpass, NTChannel$\ast$ highpass)}
    Create a Lifting object for lifting operations on the columns of 2
    \name{NTChannel}s.
    \name{lowpass} contains the low pass samples, while \name{highpass}
    contains the high pass samples. Both \name{lowpass} and \name{highpass}
    must have the same number of columns, and the number of rows of
    \name{lowpass} and \name{highpass} must differ maximum 1.

% -----------------------------------------------------------------------------

\subtitle{Plans}
    Add support for transforms of a rectangular subarea of a channel.
    emph{???}

% -----------------------------------------------------------------------------

\subtitle{Dependency \\ Graphs}
    \begin{description}
    \item[Fig.~\ref{fig:Lifting}] Inheritance dependency graph for the lifting
    class hierarchy (\emph{Lifting}).
    \end{description}

    \begin{figure}[h]\begin{center}
    \resizebox{65mm}{!}{\includegraphics{Lifting_dep.eps}}
    \caption{Inheritance dependency graph for the lifting class hierarchy
    (\emph{Lifting}).}
    \label{fig:Lifting}
    \end{center}\end{figure}

% -----------------------------------------------------------------------------

\subtitle{See Also}
    The \name{Wavelet} and \name{Channel} classes.

% -----------------------------------------------------------------------------

\subtitle{Revision}
     \rev{Lifting.C,v 4.5.2.1 1999/07/15 10:18:15 geert Exp}
     \rev{Lifting.h,v 4.3 1997/05/05 09:46:35 geert Exp}
     \rev{Lifting.inline.h,v 4.0.2.1 1999/07/20 13:18:58 geert Exp}

\end{manpage}
