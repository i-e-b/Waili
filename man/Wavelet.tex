%
%	Wavelet Classes
%
%   $Id: Wavelet.tex,v 4.6.2.4.2.1 1999/07/20 16:16:17 geert Exp $
% 
%   Copyright (C) 1996-1999 Department of Computer Science, K.U.Leuven, Belgium
% 
%   This program is free software; you can redistribute it and/or modify
%   it under the terms of the GNU General Public License as published by
%   the Free Software Foundation; either version 2 of the License, or
%   (at your option) any later version.
% 
%   This program is distributed in the hope that it will be useful,
%   but WITHOUT ANY WARRANTY; without even the implied warranty of
%   MERCHANTABILITY or FITNESS FOR A PARTICULAR PURPOSE.  See the
%   GNU General Public License for more details.
% 
%   You should have received a copy of the GNU General Public License
%   along with this program; if not, write to the Free Software
%   Foundation, Inc., 59 Temple Place, Suite 330, Boston, MA  02111-1307  USA
%

\begin{manpage}{\libtitle}{Wavelet}{$ $Revision: 4.6.2.4.2.1 $ $}

\subtitle{Name}
    Wavelet --- Integer wavelet transforms using the \name{Lifting Scheme}

% -----------------------------------------------------------------------------

\subtitle{Description}
    The basic operational step of a wavelet transform is a filter bank with 2
    kinds of filters: a low pass and a high pass filter. These 2 filters depend
    on the type of wavelet. In a wavelet transform the filter operations are
    performed iteratively on the low pass part of a signal.

    The two-dimensional wavelet transform uses the same algorithm, applied to
    both the rows and the columns of a matrix. One can consider the wavelet
    transform as a `black box' operation: a matrix is transformed into another
    matrix, its wavelet representation.

    Here the filter operations are performed in integer math using techniques
    based on the \name{Lifting Scheme}.  The sequence of Lifting steps is
    called a `Cake Walk' and strongly depends on the wavelet type.

% -----------------------------------------------------------------------------

\subtitle{Declaration}
    \needsinclude{Wavelet.h}

    \name{Wavelet} is an abstract base class. No instances can be declared.
    Different wavelet filters are implemented through inheritance.

% -----------------------------------------------------------------------------

\subtitle{Public \\ Operations}

\function{int GetGStart()\ const}
\function{int GetGEnd()\ const}
\function{int GetHStart()\ const}
\function{int GetHEnd()\ const}
    Get the start respectively end position of the high pass (`G') respectively
    low pass(`H') filter.

\function{int GetShiftL(void)\ const}
\function{int GetShiftH(void)\ const}
    Get the number of steps (in base-$\sqrt{2}$!) the coefficients of the low
    pass respectively high pass subband have to be shifted to the left to
    obtain their real values.

\function{u8 GetID(void)\ const}
    Get the unique private ID for this type of wavelet.

% -----------------------------------------------------------------------------

\subtitle{Virtual \\ Operations}

\function{Wavelet$\ast$ Clone(void)\ const}
    Make a copy of the current wavelet filter.

\function{void CakeWalk(Lifting\& lifting)\ const}
    Perform a `Cake Walk' operation on the lifting object \name{lifting}.

\function{void ICakeWalk(Lifting\& lifting)\ const}
    Perform an inverse `Cake Walk' operation on the lifting object
    \name{lifting}.

% -----------------------------------------------------------------------------

\subtitle{Static \\ Operations}

\function{Wavelet$\ast$ CreateCDF(u\_int np, u\_int nd)}
    Create a \name{Wavelet} object for some wavelet filters of the biorthogonal
    Cohen-Daubechies-Feauveau family. \name{np} and \name{nd} are the numbers
    of vanishing moments for the primal respectively dual wavelet function. The
    following wavelet bases are available. Table entries are in the form
    (\name{np}, \name{nd}):
    \begin{center}\begin{tabular}{llllll}
    $(1, 1)$ & $(1, 3)$ & $(1, 5)$ & $(2, 2)$ & $(2, 4)$ & $(2, 6)$ \\
    $(4, 2)$ & $(4, 4)$ & $(4, 6)$ \\
    \end{tabular}\end{center}
    Note that $(1, 1)$ is the Haar basis, and $(1, 3)$ is the wavelet basis
    used by Ricoh's CREW.

    $(0, 0)$ is used for the lazy wavelet filter.

\function{Wavelet$\ast$ CreateFromID(u8 id)}
    Create a \name{Wavelet} object that corresponds to the unique private ID
    \name{id}.

% -----------------------------------------------------------------------------

\separator

\subtitle{Name}
    Wavelet\_Lazy --- \name{Lazy} integer wavelet transform using the
    		      \name{Lifting Scheme}

% -----------------------------------------------------------------------------

\subtitle{Declaration}

\function{Wavelet\_Lazy()}
    Create a \name{Wavelet} object for the lazy integer wavelet transform.

% -----------------------------------------------------------------------------

\separator

\subtitle{Name}
    Wavelet\_CDF\_x\_y --- \name{Cohen-Daubechies-Feauveau} (x, y) integer
			   wavelet transforms using the \name{Lifting Scheme}

% -----------------------------------------------------------------------------

\subtitle{Declaration}

\function{Wavelet\_CDF\_1\_1()}
\function{Wavelet\_CDF\_1\_3()}
\function{Wavelet\_CDF\_1\_5()}
\function{Wavelet\_CDF\_2\_2()}
\function{Wavelet\_CDF\_2\_4()}
\function{Wavelet\_CDF\_2\_6()}
\function{Wavelet\_CDF\_4\_2()}
\function{Wavelet\_CDF\_4\_4()}
\function{Wavelet\_CDF\_4\_6()}
    Create a \name{Wavelet} object for the Cohen-Daubechies-Feauveau (x, y)
    integer wavelet transform.

% -----------------------------------------------------------------------------

\subtitle{Note}
    Internally there also exist the classes \name{LiftCoefI\_CDF\_?\_?}.

% -----------------------------------------------------------------------------

\separator

\subtitle{Name}
    Wavelet\_CRF\_13\_7, Wavelet\_SWE\_13\_7  --- Integer wavelet transforms
						  using the \name{Lifting
						  Scheme} for some more
						  wavelets used by JPEG2000.

% -----------------------------------------------------------------------------

\subtitle{Declaration}

\function{Wavelet\_CRF\_13\_7()}
\function{Wavelet\_SWE\_13\_7()}
    Create a \name{Wavelet} object for the CRF (13, 7) and SWE (13, 7) integer
    wavelet transforms.

% -----------------------------------------------------------------------------

\separator

\subtitle{Dependency \\ Graphs}
    \begin{description}
    \item[Fig.~\ref{fig:Wavelet}] Inheritance dependency graph for the Wavelet
    class hierarchy (\emph{Wavelet}).
    \end{description}

    \begin{figure}[ht]\begin{center}
    \resizebox{\textwidth}{!}{\includegraphics{Wavelet_dep.eps}}
    \caption{Inheritance dependency graph for the Wavelet class hierarchy
    (\emph{Wavelet}).}
    \label{fig:Wavelet}
    \end{center}\end{figure}

% -----------------------------------------------------------------------------

\subtitle{See Also}
    The \name{Lifting} and \name{Channel} classes.

% -----------------------------------------------------------------------------

\subtitle{Revision}
     \rev{Wavelet.C,v 4.1.2.3 1999/04/15 12:26:44 geert Exp}
     \rev{Wavelet.h,v 4.1.2.4 1999/04/15 12:26:48 geert Exp}
     \rev{Wavelet\_CDF\_1\_x.C,v 4.1 1997/05/05 09:42:33 geert Exp}
     \rev{Wavelet\_CDF\_2\_x.C,v 4.2.2.1 1999/03/16 15:05:38 geert Exp}
     \rev{Wavelet\_CDF\_4\_x.C,v 4.2.2.1 1999/03/16 15:05:39 geert Exp}
     \rev{Wavelet\_JPEG2000.C,v 5.1.2.1 1999/04/15 10:06:05 geert Exp}

\end{manpage}
