%
%	Overview
%
%   $Id: Overview.tex,v 4.1.4.1 1999/07/20 16:16:16 geert Exp $
% 
%   Copyright (C) 1996-1999 Department of Computer Science, K.U.Leuven, Belgium
% 
%   This program is free software; you can redistribute it and/or modify
%   it under the terms of the GNU General Public License as published by
%   the Free Software Foundation; either version 2 of the License, or
%   (at your option) any later version.
% 
%   This program is distributed in the hope that it will be useful,
%   but WITHOUT ANY WARRANTY; without even the implied warranty of
%   MERCHANTABILITY or FITNESS FOR A PARTICULAR PURPOSE.  See the
%   GNU General Public License for more details.
% 
%   You should have received a copy of the GNU General Public License
%   along with this program; if not, write to the Free Software
%   Foundation, Inc., 59 Temple Place, Suite 330, Boston, MA  02111-1307  USA
%

\section{Overview of \libname}

The wavelet transform library consists of the following parts:
\begin{description}
\item[Blit] Low-level block operations
\item[Channel] Generic channel class
\item[Color] Various color representations
\item[ColorSpace] Color spaces and color space conversions
\item[Compiler] Compiler dependent definitions
\item[Image] Generic image class
\item[Lifting] Lifting steps for the Lifting Scheme
\item[Stream] Input/output with support for compression
\item[Timer] Measurement of execution times
\item[Types] Platform independent type definitions
\item[Util] Utility routines
\item[Wavelet] Wavelet transforms using the Lifting Scheme
\end{description}

\textbf{Note: Currently only \name{Image}, \name{Channel} and some parts of
\name{Wavelet} (\name{CreateCDF()}) are of general interest to application
programmers.  The other parts are only used internally or aren't completely
finished yet (\name{Color}, \name{ColorSpace}).}

\section{Manual pages}

